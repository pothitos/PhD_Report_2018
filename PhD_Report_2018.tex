\documentclass[a4paper]{article}

\usepackage{fontspec}
\setmainfont{DejaVu Serif}
\setsansfont{DejaVu Sans}

\usepackage{microtype}
\usepackage[english]{babel}

\usepackage{graphicx}

\usepackage{hyperref}
\hypersetup{
  pdfauthor = {Nikolaos Pothitos},
  pdftitle = {Annual Report for the PhD Thesis "Constraint
              Programming: Algorithms and Systems"}
}

\begin{document}

\hyphenrules{dumylang}

\includegraphics[width=3em]{athena}
\textbf{Εθνικό και Καποδιστριακό Πανεπιστήμιο Αθηνών \\
        Τμήμα Πληροφορικής και Τηλεπικοινωνιών}

\vspace{1em}

\noindent
Αριθμός Πρωτοκόλλου:

\noindent
Πρωτοκολλήθηκε την:

\vspace{1.5em}

\begin{center}
  \textbf{Υπόμνημα Ετήσιας Προόδου Διδακτορικής Διατριβής \\
          Ακαδημαϊκό Έτος 2017–2018}
\end{center}

\vspace{1em}

\noindent
Επώνυμο: Ποθητός \\
Όνομα: Νικόλαος \\
Αριθμός Μητρώου: Δ541 \\
Επιβλέπων Καθηγητής: Παναγιώτης Σταματόπουλος \\
Σύμβουλος Καθηγητής: Κωνσταντίνος Χαλάτσης \\
Σύμβουλος Καθηγητής: Σταύρος Κολλιόπουλος

\begin{center}
  (Έκταση υπομνήματος 500 έως 1000 λέξεις)
\end{center}

Η έκθεση αναφέρεται στο προηγούμενο ακαδημαϊκό έτος και
περιλαμβάνει τα προβλήματα τα οποία έχει μελετήσει ο
Υποψήφιος Διδάκτωρ, τα αποτελέσματα του έργου του, πιθανές
δημοσιεύσεις και οτιδήποτε άλλο μπορεί να χαρακτηρίσει την
πρόοδό του.

\vspace{1em}

Κατά το ακαδημαϊκό έτος 2017–2018 δημοσιεύσαμε στον
27\textsuperscript{ο} τόμο του περιοδικού IJAIT μία
επέκταση~\cite{Pothitos2018} της εργασίας που παρουσιάσαμε
στο συνέδριο SETN 2016~\cite{Pothitos2016-PoPS}. Ακολουθεί
μία σύντομη εισαγωγή στο άρθρο~\cite{Pothitos2018} στα
Αγγλικά.

In the late 1990s, \emph{Constrained Programming} (CP)
promised to separate the declaration of a problem from the
process to solve it. This work attempts to serve this
direction, by implementing and presenting a modular way to
define \emph{search methods} that seek solutions to
arbitrary \emph{Constraint Satisfaction Problems} (CSPs).
The user just declares their CSP, and it can be solved using
a portfolio of search methods already in place. Apart from
the pluggable search methods framework for any CSP, we also
introduce pluggable \emph{heuristics} for our search
methods. We found an efficient stochastic heuristics'
paradigm that smoothly combines randomness with normal
heuristics. We consider a factor of \emph{disobedience} to
normal heuristics, and we fine-tune it each time, according
to our estimation of normal heuristics' reliability
(confidence). We prove mathematically that while the
disobedience factor decreases, the stochastic heuristics
approximate deterministic normal heuristics. Our algebraic
evidence is supported by empirical evaluations on real life
problems: A new search method, namely \textsc{PoPS}, that
exploits this heuristics' paradigm, can outperform regular
well-known constructive search methods.

The initial contribution of this work is the provision of an
interface that everyone can use to define their search
methods. Apart from easing the declaration of custom search
methods, we elaborated on the algorithm behind the scenes
supporting our interface in an open source solver.

In this work, we go one step further and allow the
user\slash programmer to state their own \emph{search
methods} that can apply to any CSP. We found a framework
where the user can compose their search methods out of
conjunctive and disjunctive goals.

We also presented a well-founded paradigm to exploit both
stochastic and deterministic heuristics. Empirical
evaluations showed that our hybrid approach can produce
better results than fully random or fully deterministic
methodologies.

In order to achieve this, we approached and used heuristics
as a \emph{confidence} and \emph{reliability} measure. By
exploiting these heuristic semantics, we were able to
produce a new efficient search method, namely \textsc{PoPS},
that can outperform other methodologies. In general, our
proposed framework gives the opportunity to exploit ``on the
fly'' whichever heuristic confidence fluctuations occur.

In summary, the contributions of the paper are
\begin{itemize}
  \item the foundation of a modular \emph{search methods
        framework} for Constraint Programming,
  \item the introduction of a \emph{confidence} factor into
        regular heuristics and their gradual randomization
        when we aren't confident about them, in order to
        make them more flexible, and finally
  \item the implementation of an efficient \emph{new search
        method} \textsc{PoPS} that exploits heuristics as
        values\slash evaluations---as they are---and not
        simply as ranks of possible choices.
\end{itemize}

\bibliographystyle{alpha}
\bibliography{bibliography}

\vspace{1em}

\begin{center}
  Αθήνα, 11/10/2018

  \vspace{4em}

  Υπογραφή Υποψήφιου Διδάκτορα
\end{center}

\end{document}
